\documentclass{article}
\usepackage{amsmath}
\usepackage{amssymb}
\usepackage{amsthm}
\newtheorem{example}{Example}

\begin{document}

\begin{example}
$0=0$.
\end{example}

\begin{example}
$1+1=2$.
\end{example}

\begin{example}
$1\cdot 1=1$.
\end{example}

\begin{example}
$1\times 1=1$.
\end{example}

\begin{example}
$\frac{1}{2}\times 2=1$.
\end{example}

\begin{example}
$0<1$.
\end{example}

\begin{example}
$0\ne 1$.
\end{example}

\begin{example}
Let $x\in\mathbb{R}$. Then $x=x$.
\end{example}

\begin{example}
Let $x\in\mathbb{R}$. Then $x-x=0$.
\end{example}

\begin{example}
Let $x\in\mathbb{R}$. Then $x+x=2x$.
\end{example}

\begin{example}
Let $x\in\mathbb{R}$. Then $x^2\geq 0$.
\end{example}

\begin{example}
Let $x\in\mathbb{R}$. Assume $x\geq 1$. Then $x-1\geq 0$.
\end{example}

\begin{example}
Let $x\in\mathbb{R}$. Then $2(1+x)=2+2x$.
\end{example}

\begin{example}
Let $x\in\mathbb{R}$. Then $(1+x)x=x+x^2$.
\end{example}

\begin{example}
Let $x\in\mathbb{R}$. Then $(1+x)(1+x)=1+2x+x^2$.
\end{example}

\begin{example}
Let $x\in\mathbb{R}$. Let $y\in\mathbb{R}$. Then $(1+x)(1+y)=1+x+y+xy$.
\end{example}

\begin{example}
Let $x\in\mathbb{R}$. Let $y\in\mathbb{R}$. Then ${(x+y)}^2=x^2+2xy+y^2$.
\end{example}

\begin{example}
Let $x\in\mathbb{R}$. Let $y\in\mathbb{R}$. Then ${(x+y)}^3=x^3+3x^2y+3xy^2+y^3$.
\end{example}

\begin{example}
Let $x\in\mathbb{R}$. Let $y\in\mathbb{R}$. Then ${(x+y)}^4=x^4+4x^3y+6x^2y^2+4xy^3+y^4$.
\end{example}

\begin{example}
Let $x\in\mathbb{R}$. Let $y\in\mathbb{R}$. Then ${(x+y)}^5=x^5+5x^4y+10x^3y^2+10x^2y^3+5xy^4+y^5$.
\end{example}

\begin{example}
Let $x\in\mathbb{R}$. Let $y\in\mathbb{R}$. Then ${(x+y)}^6=x^6+6x^5y+15x^4y^2+20x^3y^3+15x^2y^4+6xy^5+y^6$.
\end{example}

\begin{example}
Let $x\in\mathbb{R}$. Let $y\in\mathbb{R}$. Then ${(x+y)}^7=x^7+7x^6y+21x^5y^2+35x^4y^3+35x^3y^4+21x^2y^5+7xy^6+y^7$.
\end{example}

\begin{example}
Let $x\in\mathbb{R}$. Let $y\in\mathbb{R}$. Then ${(x+y)}^8=x^8+8x^7y+28x^6y^2+56x^5y^3+70x^4y^4+56x^3y^5+28x^2y^6+8xy^7+y^8$.
\end{example}

\begin{example}
Let $x\in\mathbb{R}$. Let $y\in\mathbb{R}$. Then ${(x+y)}^9=x^9+9x^8y+36x^7y^2+84x^6y^3+126x^5y^4+126x^4y^5+84x^3y^6+36x^2y^7+9xy^8+y^9$.
\end{example}

\begin{example}
Let $x\in\mathbb{R}$. Then ${(x^2+1)}^2=x^4+2x^2+1$.
\end{example}

\begin{example}
Let $x\in\mathbb{R}$. Let $y\in\mathbb{R}$. Then ${(x^2+y^2)}^2=x^4+2x^2y^2+y^4$.
\end{example}

\begin{example}
Let $x\in\mathbb{R}$. Let $n\in\mathbb{N}$. Then ${(x^n+1)}^2=x^{2n}+2x^n+1$.
\end{example}

\begin{example}
Let $x\in\mathbb{R}$. Let $y\in\mathbb{R}$. Let $n\in\mathbb{N}$. Then ${(x^n+y^n)}^2=x^{2n}+2x^ny^n+y^{2n}$.
\end{example}

\begin{example}
Let $x\in\mathbb{R}$. Let $n\in\mathbb{N}$. Then ${(x^{n^2}+1)}^2=x^{2n^2}+2x^{n^2}+1$.
\end{example}

\begin{example}
Let $x\in\mathbb{R}$. Let $n\in\mathbb{N}$. Then ${(x^{2n}+1)}^2=x^{4n}+2x^{2n}+1$.
\end{example}

\begin{example}
    $1000340282366920938463463374607431768211456=1000340282366920938463463374607431768211456$.
\end{example}

\begin{example}
Let $x\in\mathbb{R}$. Let $y\in\mathbb{R}$. Then $x+y=y+x$.
\end{example}

\begin{example}
Let $x\in\mathbb{R}$. Assume $x=1$. Then $x=1$.
\end{example}

\begin{example}
Let $x=1$. Then $x=1$.
\end{example}

\begin{example}
Let $x=1$. Then $x>0$.
\end{example}

\begin{example}
Let $x=1$. Let $y=1$. Let $z=2$. Then $x+y=z$.
\end{example}

\begin{example}
Let $x\in\mathbb{R}$. Assume $x>0$. Then $x>0$.
\end{example}

\begin{example}
Let $x\in\mathbb{R}$. Assume $x>1$. Then $x>0$.
\end{example}

\begin{example}
Let $x\in\mathbb{R}$. Assume $x>1$. Then $x\ge 1$.
\end{example}

\begin{example}
Let $x\in\mathbb{R}$. Assume $x\ge 1$. Then $x\ge 0$.
\end{example}

\begin{example}
Let $x\in\mathbb{R}$. Assume $x\ge 1$. Then $x> 0$.
\end{example}

\begin{example}
Let $x\in\mathbb{R}$. Assume $x<1$. Then $x<2$.
\end{example}

% \begin{example}
% Let $x\in\mathbb{R}$. Assume $x<1$. Then $x\le 2$.
% \end{example}

\end{document}
