\documentclass{article}
\usepackage{amsmath}
\usepackage{amssymb}
\usepackage{amsthm}
\newtheorem{example}{Example}

\begin{document}

\begin{example}
$0=0$.
\end{example}

\begin{example}
$1+1=2$.
\end{example}

\begin{example}
$1\cdot 1=1$.
\end{example}

\begin{example}
$0<1$.
\end{example}

\begin{example}
$0\ne 1$.
\end{example}

\begin{example}
Let $x\in\mathbb{R}$. Then $x=x$.
\end{example}

\begin{example}
Let $x\in\mathbb{R}$. Then $x-x=0$.
\end{example}

\begin{example}
Let $x\in\mathbb{R}$. Then $x+x=2x$.
\end{example}

\begin{example}
Let $x\in\mathbb{R}$. Then $x^2\geq 0$.
\end{example}

\begin{example}
Let $x\in\mathbb{R}$. Assume $x\geq 1$. Then $x-1\geq 0$.
\end{example}

\begin{example}
Let $x\in\mathbb{R}$. Then $2(1+x)=2+2x$.
\end{example}

\begin{example}
Let $x\in\mathbb{R}$. Then $(1+x)x=x+x^2$.
\end{example}

\end{document}
