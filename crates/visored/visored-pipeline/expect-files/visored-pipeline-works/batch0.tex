\documentclass{article}
\usepackage{amsmath}
\usepackage{amssymb}
\usepackage{fvextra}
\usepackage{tcolorbox}
\usepackage{listings}
\usepackage{amsthm}
\newtheorem{example}{Example}
\fvset{breaklines=true}

\begin{document}
\lstdefinelanguage{LaTeX}{
    breaklines=true,
    basicstyle=\ttfamily\normalsize,
    basewidth={0.6em,0.45em},
    keepspaces=true,
    morekeywords={
            \begin,\end,\usepackage,\documentclass,\maketitle,
            \title,\author,\section,\subsection,\textbf
        },
    alsoletter={\\},      % Treat backslash as part of keywords
    morecomment=[l]{\%},  % Line comment starts with '%'
    morestring=[b]",      % Strings in double quotes
    sensitive=true        % Case-sensitive
}



\begin{example}
Problem:
\begin{tcolorbox}[colback=yellow!10, width=\linewidth]
Prove that for all real numbers $a$ and $b$:
    $$(a+b)^2 \geq 0.$$
\end{tcolorbox}

Simplified proof:
\begin{tcolorbox}[colback=blue!10, width=\linewidth]
Since the square of any real number is non-negative, $(a+b)^2 \ge 0$.
\end{tcolorbox}
\end{example}

Elaborated proof:
\begin{tcolorbox}[colback=green!10, width=\linewidth]
Since the square of any real number is non-negative, $(a+b)^2 = (a+b)(a+b) = a^2 + ab + ba + b^2 = a^2 + 2ab + b^2 \ge 0$.
\end{tcolorbox}

Regularized proof:
\begin{tcolorbox}[colback=red!10, width=\linewidth]
It's enough to prove that $(a+b)^2 \geq 0$. We have $(a+b)^2 = a^2 + 2ab + b^2$ by algebra. We have $a^2 \ge 0$ by the fact that the square of any real number is non-negative. We have $b^2 \ge 0$ by the fact that the square of any real number is non-negative. We have $a^2 + 2ab + b^2 \ge 0$ by the fact that the sum of non-negative numbers is non-negative. We have $(a+b)^2 \geq 0$ by algebra.
\end{tcolorbox}



\begin{example}
Problem:
\begin{tcolorbox}[colback=yellow!10, width=\linewidth]
Prove that for any positive real numbers $x$ and $y$:
    $$\frac{x+y}{2} \geq \sqrt{xy}.$$
\end{tcolorbox}

Simplified proof:
\begin{tcolorbox}[colback=blue!10, width=\linewidth]
Since $(x-y)^2 \ge 0$ for all $x, y \in \mathbb{R}$, we have $x^2 - 2xy + y^2 \ge 0$. Then $(x+y)^2 = x^2 + 2xy + y^2 \ge 4xy$, so $x+y \ge 2\sqrt{xy}$, and $\frac{x+y}{2} \ge \sqrt{xy}$.
\end{tcolorbox}
\end{example}

Elaborated proof:
\begin{tcolorbox}[colback=green!10, width=\linewidth]
Since $(x-y)^2 \ge 0$ for all $x, y \in \mathbb{R}$, we have $x^2 - 2xy + y^2 \ge 0$. Then $(x+y)^2 = x^2 + 2xy + y^2 = (x^2 - 2xy + y^2) + 4xy \ge 4xy$, so $x+y \ge 2\sqrt{xy}$, and $\frac{x+y}{2} \ge \sqrt{xy}$.
\end{tcolorbox}

Regularized proof:
\begin{tcolorbox}[colback=red!10, width=\linewidth]
It's enough to prove that $x+y \ge 2\sqrt{xy}$. We have $(x-y)^2 \ge 0$ by definition. We have $x^2 - 2xy + y^2 \ge 0$ by algebra. We have $(x+y)^2 = x^2 + 2xy + y^2$ by algebra. We have $(x+y)^2 = (x^2 - 2xy + y^2) + 4xy$ by algebra. We have $(x+y)^2 \ge 4xy$ by algebra. We have $x+y \ge 2\sqrt{xy}$ by algebra. We have $\frac{x+y}{2} \ge \sqrt{xy}$ by algebra.
\end{tcolorbox}



\begin{example}
Problem:
\begin{tcolorbox}[colback=yellow!10, width=\linewidth]
Show that for all real numbers $a$, $b$, and $c$:
    $$a^2 + b^2 + c^2 \geq ab + bc + ca.$$
\end{tcolorbox}

Simplified proof:
\begin{tcolorbox}[colback=blue!10, width=\linewidth]
Since $(a-b)^2+(b-c)^2+(c-a)^2 = 2(a^2+b^2+c^2-ab-bc-ca) \ge 0$, we have $a^2+b^2+c^2 \ge ab+bc+ca$.
\end{tcolorbox}
\end{example}

Elaborated proof:
\begin{tcolorbox}[colback=green!10, width=\linewidth]
Since $(a-b)^2+(b-c)^2+(c-a)^2 = a^2 - 2ab + b^2 + b^2 - 2bc + c^2 + c^2 - 2ca + a^2 = 2(a^2+b^2+c^2-ab-bc-ca) \ge 0$, we have $2(a^2+b^2+c^2-ab-bc-ca) \ge 0$, which means $a^2+b^2+c^2-ab-bc-ca \ge 0$, so $a^2+b^2+c^2 \ge ab+bc+ca$.
\end{tcolorbox}

Regularized proof:
\begin{tcolorbox}[colback=red!10, width=\linewidth]
It's enough to prove that $2(a^2+b^2+c^2-ab-bc-ca) \ge 0$. We have $(a-b)^2+(b-c)^2+(c-a)^2 = 2(a^2+b^2+c^2-ab-bc-ca)$ by expansion. We have $(a-b)^2+(b-c)^2+(c-a)^2 \ge 0$ by the fact that the square of a real number is nonnegative. We have $2(a^2+b^2+c^2-ab-bc-ca) \ge 0$ by transitivity. We have $a^2+b^2+c^2-ab-bc-ca \ge 0$ by dividing by 2. We have $a^2+b^2+c^2 \ge ab+bc+ca$ by adding $ab+bc+ca$ to both sides.
\end{tcolorbox}



\begin{example}
Problem:
\begin{tcolorbox}[colback=yellow!10, width=\linewidth]
Prove that for any positive real number $x$:
    $$x + \frac{1}{x} \geq 2.$$
\end{tcolorbox}

Simplified proof:
\begin{tcolorbox}[colback=blue!10, width=\linewidth]
Since $(x-1)^2 \ge 0$ for all $x$, we have $x^2 - 2x + 1 \ge 0$. Then $x^2 + 1 \ge 2x$, so $x + \frac{1}{x} \ge 2$ for $x>0$.
\end{tcolorbox}
\end{example}

Elaborated proof:
\begin{tcolorbox}[colback=green!10, width=\linewidth]
Since $(x-1)^2 \ge 0$ for all $x$, we have $x^2 - 2x + 1 \ge 0$, which is equivalent to $x^2 + 1 \ge 2x$.  Then, for $x>0$, we can divide both sides by $x$ to get $x + \frac{1}{x} \ge 2$.
\end{tcolorbox}

Regularized proof:
\begin{tcolorbox}[colback=red!10, width=\linewidth]
It's enough to prove that $x + \frac{1}{x} \geq 2$. We have $(x-1)^2 \ge 0$ by definition of squares. We have $x^2 - 2x + 1 \ge 0$ by expanding the square. We have $x^2 + 1 \ge 2x$ by adding $2x$ to both sides. We have $x + \frac{1}{x} \ge 2$ by dividing both sides by $x$ for $x>0$.
\end{tcolorbox}



\begin{example}
Problem:
\begin{tcolorbox}[colback=yellow!10, width=\linewidth]
For positive real numbers $a$ and $b$, prove:
    $$\left(\frac{a+b}{2}\right)^2 \leq \frac{a^2+b^2}{2}.$$
\end{tcolorbox}

Simplified proof:
\begin{tcolorbox}[colback=blue!10, width=\linewidth]
Since $(a-b)^2 \ge 0$, we have $a^2 - 2ab + b^2 \ge 0$, so $a^2 + b^2 \ge 2ab$.  Adding $a^2 + b^2$ to both sides gives $2(a^2+b^2) \ge (a+b)^2$. Dividing by 4 yields $\frac{a^2+b^2}{2} \ge \left(\frac{a+b}{2}\right)^2$.
\end{tcolorbox}
\end{example}

Elaborated proof:
\begin{tcolorbox}[colback=green!10, width=\linewidth]
Since $(a-b)^2 \ge 0$, we have $a^2 - 2ab + b^2 \ge 0$, so $a^2 + b^2 \ge 2ab$.  Adding $a^2 + b^2$ to both sides gives $2a^2 + 2b^2 \ge a^2 + 2ab + b^2 = (a+b)^2$. Dividing by 4 yields $\frac{2a^2 + 2b^2}{4} \ge \frac{(a+b)^2}{4}$, which simplifies to $\frac{a^2+b^2}{2} \ge \left(\frac{a+b}{2}\right)^2$.
\end{tcolorbox}

Regularized proof:
\begin{tcolorbox}[colback=red!10, width=\linewidth]
We have $(a-b)^2 \ge 0$ by definition. We have $a^2 - 2ab + b^2 \ge 0$ by expanding $(a-b)^2$. We have $a^2 + b^2 \ge 2ab$ by adding $2ab$ to both sides. We have $2a^2 + 2b^2 \ge a^2 + 2ab + b^2$ by adding $a^2 + b^2$ to both sides. We have $2a^2 + 2b^2 \ge (a+b)^2$ by factoring. We have $\frac{2a^2 + 2b^2}{4} \ge \frac{(a+b)^2}{4}$ by dividing by 4. We have $\frac{a^2+b^2}{2} \ge \left(\frac{a+b}{2}\right)^2$ by simplifying.
\end{tcolorbox}

\end{document}