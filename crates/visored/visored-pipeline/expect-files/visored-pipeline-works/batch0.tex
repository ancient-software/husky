\documentclass{article}
\usepackage{amsmath}
\usepackage{amssymb}
\usepackage{fvextra}
\usepackage{tcolorbox}
\usepackage{listings}
\fvset{breaklines=true}

\begin{document}
\lstdefinelanguage{LaTeX}{
    breaklines=true,
    basicstyle=\ttfamily\normalsize,
    basewidth={0.6em,0.45em},
    keepspaces=true,
    morekeywords={
            \begin,\end,\usepackage,\documentclass,\maketitle,
            \title,\author,\section,\subsection,\textbf
        },
    alsoletter={\\},      % Treat backslash as part of keywords
    morecomment=[l]{\%},  % Line comment starts with '%'
    morestring=[b]",      % Strings in double quotes
    sensitive=true        % Case-sensitive
}



Simplified solution:
\begin{tcolorbox}[colback=blue!10, width=\linewidth]
Let $x = a+b$.  Since $a$ and $b$ are real numbers, $x$ is a real number.  Then $x^2 \ge 0$, so $(a+b)^2 \ge 0$.

\end{tcolorbox}



Simplified solution:
\begin{tcolorbox}[colback=blue!10, width=\linewidth]
Let $x = a+b$.  Since $a$ and $b$ are real numbers, $x$ is a real number.  Then $x^2 \ge 0$, so $(a+b)^2 \ge 0$.

\end{tcolorbox}



Simplified solution:
\begin{tcolorbox}[colback=blue!10, width=\linewidth]
$(\sqrt{x} - \sqrt{y})^2 \ge 0$
$x - 2\sqrt{xy} + y \ge 0$
$x + y \ge 2\sqrt{xy}$
$\frac{x+y}{2} \ge \sqrt{xy}$
\end{tcolorbox}



Simplified solution:
\begin{tcolorbox}[colback=blue!10, width=\linewidth]
$(\sqrt{x} - \sqrt{y})^2 \ge 0$
$x - 2\sqrt{xy} + y \ge 0$
$x + y \ge 2\sqrt{xy}$
$\frac{x+y}{2} \ge \sqrt{xy}$
\end{tcolorbox}



Simplified solution:
\begin{tcolorbox}[colback=blue!10, width=\linewidth]
\begin{align*} 0 &\le (a-b)^2 + (b-c)^2 + (c-a)^2 \\ &= a^2 - 2ab + b^2 + b^2 - 2bc + c^2 + c^2 - 2ac + a^2 \\ &= 2a^2 + 2b^2 + 2c^2 - 2ab - 2bc - 2ac \\ 2ab + 2bc + 2ac &\le 2a^2 + 2b^2 + 2c^2 \\ ab + bc + ac &\le a^2 + b^2 + c^2\end{align*}
\end{tcolorbox}



Simplified solution:
\begin{tcolorbox}[colback=blue!10, width=\linewidth]
\begin{align*} 0 &\le (a-b)^2 + (b-c)^2 + (c-a)^2 \\ &= a^2 - 2ab + b^2 + b^2 - 2bc + c^2 + c^2 - 2ac + a^2 \\ &= 2a^2 + 2b^2 + 2c^2 - 2ab - 2bc - 2ac \\ 2ab + 2bc + 2ac &\le 2a^2 + 2b^2 + 2c^2 \\ ab + bc + ac &\le a^2 + b^2 + c^2\end{align*}
\end{tcolorbox}



Simplified solution:
\begin{tcolorbox}[colback=blue!10, width=\linewidth]
Since $x$ is a positive real number, we can use the AM-GM inequality:
\[ \frac{x + \frac{1}{x}}{2} \ge \sqrt{x \cdot \frac{1}{x}} = \sqrt{1} = 1 \]
Multiplying both sides by 2, we get:
\[ x + \frac{1}{x} \ge 2 \]
Alternatively, consider $( \sqrt{x} - \frac{1}{\sqrt{x}})^2 \ge 0$. Expanding this gives:
\[ x - 2 + \frac{1}{x} \ge 0 \]
\[ x + \frac{1}{x} \ge 2 \]
\end{tcolorbox}



Simplified solution:
\begin{tcolorbox}[colback=blue!10, width=\linewidth]
Since $x$ is a positive real number, we can use the AM-GM inequality:
\[ \frac{x + \frac{1}{x}}{2} \ge \sqrt{x \cdot \frac{1}{x}} = \sqrt{1} = 1 \]
Multiplying both sides by 2, we get:
\[ x + \frac{1}{x} \ge 2 \]
Alternatively, consider $( \sqrt{x} - \frac{1}{\sqrt{x}})^2 \ge 0$. Expanding this gives:
\[ x - 2 + \frac{1}{x} \ge 0 \]
\[ x + \frac{1}{x} \ge 2 \]
\end{tcolorbox}



Simplified solution:
\begin{tcolorbox}[colback=blue!10, width=\linewidth]
\[
\left(\frac{a+b}{2}\right)^2 \leq \frac{a^2+b^2}{2}
\]
\[
\frac{(a+b)^2}{4} \leq \frac{a^2+b^2}{2}
\]
\[
(a+b)^2 \leq 2(a^2+b^2)
\]
\[
a^2 + 2ab + b^2 \leq 2a^2 + 2b^2
\]
\[
0 \leq a^2 - 2ab + b^2
\]
\[
0 \leq (a-b)^2
\]
This is true for all real numbers $a$ and $b$, since the square of a real number is always non-negative.

\end{tcolorbox}



Simplified solution:
\begin{tcolorbox}[colback=blue!10, width=\linewidth]
\[
\left(\frac{a+b}{2}\right)^2 \leq \frac{a^2+b^2}{2}
\]
\[
\frac{(a+b)^2}{4} \leq \frac{a^2+b^2}{2}
\]
\[
(a+b)^2 \leq 2(a^2+b^2)
\]
\[
a^2 + 2ab + b^2 \leq 2a^2 + 2b^2
\]
\[
0 \leq a^2 - 2ab + b^2
\]
\[
0 \leq (a-b)^2
\]
This is true for all real numbers $a$ and $b$, since the square of a real number is always non-negative.

\end{tcolorbox}

\end{document}