\documentclass{article}
\usepackage{amsmath}
\usepackage{amssymb}
\usepackage{fvextra}
\usepackage{tcolorbox}
\usepackage{listings}
\fvset{breaklines=true}

\begin{document}
\lstdefinelanguage{LaTeX}{
    breaklines=true,
    basicstyle=\ttfamily\normalsize,
    basewidth={0.6em,0.45em},
    keepspaces=true,
    morekeywords={
            \begin,\end,\usepackage,\documentclass,\maketitle,
            \title,\author,\section,\subsection,\textbf
        },
    alsoletter={\\},      % Treat backslash as part of keywords
    morecomment=[l]{\%},  % Line comment starts with '%'
    morestring=[b]",      % Strings in double quotes
    sensitive=true        % Case-sensitive
}



Simplified solution:
\begin{tcolorbox}[colback=blue!10, width=\linewidth]
Since the square of any real number is non-negative, $(a+b)^2 \ge 0$ for all real numbers $a$ and $b$.
\end{tcolorbox}



Simplified solution:
\begin{tcolorbox}[colback=blue!10, width=\linewidth]
Since the square of any real number is non-negative, $(a+b)^2 \ge 0$ for all real numbers $a$ and $b$.
\end{tcolorbox}



Simplified solution:
\begin{tcolorbox}[colback=blue!10, width=\linewidth]
Since $(x-y)^2 \ge 0$, we have $x^2 - 2xy + y^2 \ge 0$.  Then $x^2 + 2xy + y^2 \ge 4xy$, so $(x+y)^2 \ge 4xy$. Taking the square root of both sides (since $x, y > 0$), we get $x+y \ge 2\sqrt{xy}$. Dividing by 2 gives $\frac{x+y}{2} \ge \sqrt{xy}$.
\end{tcolorbox}



Simplified solution:
\begin{tcolorbox}[colback=blue!10, width=\linewidth]
Since $(x-y)^2 \ge 0$, we have $x^2 - 2xy + y^2 \ge 0$.  Then $x^2 + 2xy + y^2 \ge 4xy$, so $(x+y)^2 \ge 4xy$. Taking the square root of both sides (since $x, y > 0$), we get $x+y \ge 2\sqrt{xy}$. Dividing by 2 gives $\frac{x+y}{2} \ge \sqrt{xy}$.
\end{tcolorbox}



Simplified solution:
\begin{tcolorbox}[colback=blue!10, width=\linewidth]
\begin{align*} 0 &\le (a-b)^2 + (b-c)^2 + (c-a)^2 \\ &= a^2 - 2ab + b^2 + b^2 - 2bc + c^2 + c^2 - 2ac + a^2 \\ &= 2a^2 + 2b^2 + 2c^2 - 2ab - 2bc - 2ac \\ 2(ab+bc+ca) &\le 2(a^2+b^2+c^2) \\ ab+bc+ca &\le a^2+b^2+c^2 \end{align*}
\end{tcolorbox}



Simplified solution:
\begin{tcolorbox}[colback=blue!10, width=\linewidth]
\begin{align*} 0 &\le (a-b)^2 + (b-c)^2 + (c-a)^2 \\ &= a^2 - 2ab + b^2 + b^2 - 2bc + c^2 + c^2 - 2ac + a^2 \\ &= 2a^2 + 2b^2 + 2c^2 - 2ab - 2bc - 2ac \\ 2(ab+bc+ca) &\le 2(a^2+b^2+c^2) \\ ab+bc+ca &\le a^2+b^2+c^2 \end{align*}
\end{tcolorbox}



Simplified solution:
\begin{tcolorbox}[colback=blue!10, width=\linewidth]
Since $x$ is a positive real number, by the AM-GM inequality, we have $\frac{x + \frac{1}{x}}{2} \ge \sqrt{x \cdot \frac{1}{x}} = \sqrt{1} = 1$. Multiplying both sides by 2 gives $x + \frac{1}{x} \ge 2$. Equality holds if and only if $x = \frac{1}{x}$, which implies $x = 1$.
\end{tcolorbox}



Simplified solution:
\begin{tcolorbox}[colback=blue!10, width=\linewidth]
Since $x$ is a positive real number, by the AM-GM inequality, we have $\frac{x + \frac{1}{x}}{2} \ge \sqrt{x \cdot \frac{1}{x}} = \sqrt{1} = 1$. Multiplying both sides by 2 gives $x + \frac{1}{x} \ge 2$. Equality holds if and only if $x = \frac{1}{x}$, which implies $x = 1$.
\end{tcolorbox}



Simplified solution:
\begin{tcolorbox}[colback=blue!10, width=\linewidth]
\begin{align*}
\frac{a^2+b^2}{2} - \left(\frac{a+b}{2}\right)^2 &= \frac{a^2+b^2}{2} - \frac{a^2+2ab+b^2}{4} \\
&= \frac{2a^2+2b^2 - a^2 - 2ab - b^2}{4} \\
&= \frac{a^2 - 2ab + b^2}{4} \\
&= \frac{(a-b)^2}{4} \geq 0
\end{align*} 
Since $\frac{(a-b)^2}{4} \ge 0$, we have $\frac{a^2+b^2}{2} \ge \left(\frac{a+b}{2}\right)^2$.
\end{tcolorbox}



Simplified solution:
\begin{tcolorbox}[colback=blue!10, width=\linewidth]
\begin{align*}
\frac{a^2+b^2}{2} - \left(\frac{a+b}{2}\right)^2 &= \frac{a^2+b^2}{2} - \frac{a^2+2ab+b^2}{4} \\
&= \frac{2a^2+2b^2 - a^2 - 2ab - b^2}{4} \\
&= \frac{a^2 - 2ab + b^2}{4} \\
&= \frac{(a-b)^2}{4} \geq 0
\end{align*} 
Since $\frac{(a-b)^2}{4} \ge 0$, we have $\frac{a^2+b^2}{2} \ge \left(\frac{a+b}{2}\right)^2$.
\end{tcolorbox}

\end{document}