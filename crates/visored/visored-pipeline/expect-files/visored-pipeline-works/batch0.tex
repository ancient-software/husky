%!TEX TS-program = xelatex
\documentclass{article}
\usepackage{amsmath}
\usepackage{amssymb}
\usepackage{fvextra}
\usepackage{tcolorbox}
\usepackage{listings}
\usepackage{amsthm}
\usepackage{fontspec}  % For Unicode support
\setmonofont{DejaVu Sans Mono}  % For monospace/code blocks
\usepackage{unicode-math}
% \setmathfont{XITS Math}  % Or another Unicode math font
\newtheorem{example}{Example}
\fvset{breaklines=true}



\begin{document}
\lstdefinelanguage{Lean4}{
    breaklines=true,
    basicstyle=\ttfamily\normalsize,
    keepspaces=true,
    morekeywords={
        def, theorem, lemma, example, have, show, calc, let, assume, by, exact,
        sorry, obvious, Type, Prop, where, with, extends, class, instance,
        structure, inductive, mutual, coinductive, variable, variables,
        universe, universes, deriving, abbrev, partial, terminating,
        namespace, import, open, export, private, protected, public,
        noncomputable, unsafe, macro, syntax, macro_rules, set_option,
        attribute, local, scoped, section, end, match, fun, if, then, else,
        return, do, for, in, while, break, continue, try, catch, throw,
        mut, pure, opaque
    },
    morekeywords=[2]{
        ℕ, ℝ, ∈, ≥, ≤, →, ∀, ∃, ⊢, ∧, ∨, ¬, ≠, ×, ⊗, ⊕, ∘, □, ◇, ∎,
        ⟨, ⟩, ⦃, ⦄, ▸, ≈, ∼, ≡, ⌊, ⌋, ⌈, ⌉
    },
    morecomment=[l]--,     % Line comments start with --
    morestring=[b]",       % Strings in double quotes
    sensitive=true,         % Case-sensitive
    keywordstyle=\color{blue}\bfseries,      % Regular keywords in blue and bold
    keywordstyle=[2]\color{purple}\bfseries, % Special symbols in purple and bold
    commentstyle=\color{green!50!black},     % Comments in dark green
    stringstyle=\color{red},                 % Strings in red
}



\begin{example}
Problem:
\begin{tcolorbox}[colback=yellow!10, width=\linewidth]
Prove that for all real numbers $a$ and $b$:
    $$(a+b)^2 \geq 0.$$
\end{tcolorbox}

Simplified proof:
\begin{tcolorbox}[colback=blue!10, width=\linewidth]
Trivial since $(a+b)^2 \ge 0$.
\end{tcolorbox}
\end{example}

Elaborated proof:
\begin{tcolorbox}[colback=green!10, width=\linewidth]
Trivial since $(a+b)^2 \ge 0$.
\end{tcolorbox}

Regularized proof:
\begin{tcolorbox}[colback=red!10, width=\linewidth]
Let $a\in\mathbb{R}$.
Let $b\in\mathbb{R}$.
The goal is to prove ${{(a+b)}}^2 \ge 0$.
We have ${{(a+b)}}^2 \ge 0$.
\end{tcolorbox}

Lean 4 code:
\begin{tcolorbox}[colback=white!10, width=\linewidth]
\begin{lstlisting}[language=Lean4]
import Mathlib
import Obvious
open Obvious

def h(a : ℝ)(b : ℝ) : (a + b) ^ 2 ≥ 0 := by
  have h1 : (a + b) ^ 2 ≥ 0 := by obvious
  obvious

\end{lstlisting}
\end{tcolorbox}


\begin{example}
Problem:
\begin{tcolorbox}[colback=yellow!10, width=\linewidth]
Prove that for any positive real numbers $x$ and $y$:
    $$\frac{x+y}{2} \geq \sqrt{xy}.$$
\end{tcolorbox}

Simplified proof:
\begin{tcolorbox}[colback=blue!10, width=\linewidth]
Since $x$ and $y$ are positive, $(\sqrt x - \sqrt y)^2 \ge 0$, so $x+y \ge 2\sqrt{xy}$, which implies $\frac{x+y}{2} \ge \sqrt{xy}$.
\end{tcolorbox}
\end{example}

Elaborated proof:
\begin{tcolorbox}[colback=green!10, width=\linewidth]
Since $x$ and $y$ are positive, $(\sqrt x - \sqrt y)^2 = (\sqrt{x})^2 - 2\sqrt{x}\sqrt{y} + (\sqrt{y})^2= x - 2\sqrt{xy} + y \ge 0$, so $x+y \ge 2\sqrt{xy}$, which implies $\frac{x+y}{2} \ge \sqrt{xy}$.
\end{tcolorbox}

Regularized proof:
\begin{tcolorbox}[colback=red!10, width=\linewidth]
Let $x\in\mathbb{R}$. Let $y\in\mathbb{R}$.
Assume $x \ge 0$. Assume $y \ge 0$.
The goal is to prove $\frac{x+y}{2} \ge \sqrt{xy}$.
We have ${{(\sqrt x - \sqrt y)}}^2 = {{(\sqrt{x})}}^2 - 2\sqrt{x}\sqrt{y} + {{(\sqrt{y})}}^2= x - 2\sqrt{xy} + y$.
We have ${{(\sqrt x - \sqrt y)}}^2 \ge 0$ because $x$ and $y$ are positive.
We have $x - 2\sqrt{xy} + y \ge 0$.
We have $x+y \ge 2\sqrt{xy}$.
We have $\frac{x+y}{2} \ge \sqrt{xy}$.
\end{tcolorbox}

Lean 4 code:
\begin{tcolorbox}[colback=white!10, width=\linewidth]
\begin{lstlisting}[language=Lean4]
import Mathlib
import Obvious
open Obvious

def h(x : ℝ)(y : ℝ)(h1 : x ≥ 0)(h2 : y ≥ 0) : (x + y) / 2 ≥ √ (x * y) := by
  first
  | have h3 : (√ x - √ y) ^ 2 = x - 2 * √ (x * y) + y := by calc
    (√ x - √ y) ^ 2 = √ x ^ 2 - 2 * √ x * √ y + √ y ^ 2 := by obvious
    _ = x - 2 * √ (x * y) + y := by obvious
  | have h4 : x - 2 * √ (x * y) + y = (√ x - √ y) ^ 2 := by calc
    x - 2 * √ (x * y) + y = √ x ^ 2 - 2 * √ x * √ y + √ y ^ 2 := by obvious
    _ = (√ x - √ y) ^ 2 := by obvious
  have h5 : (√ x - √ y) ^ 2 ≥ 0 := by obvious
  have h6 : x - 2 * √ (x * y) + y ≥ 0 := by obvious
  have h7 : x + y ≥ 2 * √ (x * y) := by obvious
  have h8 : (x + y) / 2 ≥ √ (x * y) := by obvious
  obvious

\end{lstlisting}
\end{tcolorbox}


\begin{example}
Problem:
\begin{tcolorbox}[colback=yellow!10, width=\linewidth]
Show that for all real numbers $a$, $b$, and $c$:
    $$a^2 + b^2 + c^2 \geq ab + bc + ca.$$
\end{tcolorbox}

Simplified proof:
\begin{tcolorbox}[colback=blue!10, width=\linewidth]
We have $(a-b)^2+(b-c)^2+(c-a)^2 \ge 0$, so $a^2+b^2+c^2 \ge ab+bc+ca$.
\end{tcolorbox}
\end{example}

Elaborated proof:
\begin{tcolorbox}[colback=green!10, width=\linewidth]
We have $(a-b)^2+(b-c)^2+(c-a)^2 = a^2 - 2ab + b^2 + b^2 - 2bc + c^2 + c^2 - 2ca + a^2 = 2(a^2+b^2+c^2 - ab - bc - ca) \ge 0$, so $a^2+b^2+c^2 \ge ab+bc+ca$.
\end{tcolorbox}

Regularized proof:
\begin{tcolorbox}[colback=red!10, width=\linewidth]
Let $a\in\mathbb{R}$. Let $b\in\mathbb{R}$. Let $c\in\mathbb{R}$.
The goal is to prove $a^2+b^2+c^2 \ge ab+bc+ca$.
We have ${{(a-b)}}^2+{{(b-c)}}^2+{{(c-a)}}^2 = a^2 - 2ab + b^2 + b^2 - 2bc + c^2 + c^2 - 2ca + a^2 = 2(a^2+b^2+c^2 - ab - bc - ca)$.
We have ${{(a-b)}}^2+{{(b-c)}}^2+{{(c-a)}}^2 \ge 0$.
We have $2(a^2+b^2+c^2 - ab - bc - ca) \ge 0$.
We have $a^2+b^2+c^2 \ge ab+bc+ca$.
\end{tcolorbox}

Lean 4 code:
\begin{tcolorbox}[colback=white!10, width=\linewidth]
\begin{lstlisting}[language=Lean4]
import Mathlib
import Obvious
open Obvious

def h(a : ℝ)(b : ℝ)(c : ℝ) : a ^ 2 + b ^ 2 + c ^ 2 ≥ a * b + b * c + c * a := by
  first
  | have h1 : (a - b) ^ 2 + (b - c) ^ 2 + (c - a) ^ 2 = 2 * (a ^ 2 + b ^ 2 + c ^ 2 - a * b - b * c - c * a) := by calc
    (a - b) ^ 2 + (b - c) ^ 2 + (c - a) ^ 2 = a ^ 2 - 2 * a * b + b ^ 2 + b ^ 2 - 2 * b * c + c ^ 2 + c ^ 2 - 2 * c * a + a ^ 2 := by obvious
    _ = 2 * (a ^ 2 + b ^ 2 + c ^ 2 - a * b - b * c - c * a) := by obvious
  | have h2 : 2 * (a ^ 2 + b ^ 2 + c ^ 2 - a * b - b * c - c * a) = (a - b) ^ 2 + (b - c) ^ 2 + (c - a) ^ 2 := by calc
    2 * (a ^ 2 + b ^ 2 + c ^ 2 - a * b - b * c - c * a) = a ^ 2 - 2 * a * b + b ^ 2 + b ^ 2 - 2 * b * c + c ^ 2 + c ^ 2 - 2 * c * a + a ^ 2 := by obvious
    _ = (a - b) ^ 2 + (b - c) ^ 2 + (c - a) ^ 2 := by obvious
  have h3 : (a - b) ^ 2 + (b - c) ^ 2 + (c - a) ^ 2 ≥ 0 := by obvious
  have h4 : 2 * (a ^ 2 + b ^ 2 + c ^ 2 - a * b - b * c - c * a) ≥ 0 := by obvious
  have h5 : a ^ 2 + b ^ 2 + c ^ 2 ≥ a * b + b * c + c * a := by obvious
  obvious

\end{lstlisting}
\end{tcolorbox}


\begin{example}
Problem:
\begin{tcolorbox}[colback=yellow!10, width=\linewidth]
Prove that for any positive real number $x$:
    $$x + \frac{1}{x} \geq 2.$$
\end{tcolorbox}

Simplified proof:
\begin{tcolorbox}[colback=blue!10, width=\linewidth]
Since $x>0$, we have $(x-1)^2 \ge 0$. Thus, $x + \frac{1}{x} \ge 2$.
\end{tcolorbox}
\end{example}

Elaborated proof:
\begin{tcolorbox}[colback=green!10, width=\linewidth]
Since $x>0$, we have $(x-1)^2 = x^2 - 2x + 1 \ge 0$. Thus, $x^2 + 1 \ge 2x$, so $x + \frac{1}{x} \ge 2$.
\end{tcolorbox}

Regularized proof:
\begin{tcolorbox}[colback=red!10, width=\linewidth]
Let $x\in\mathbb{R}$.
Assume $x > 0$.
The goal is to prove $x + \frac{1}{x} \ge 2$.
We have $x>0$.
We have ${(x-1)}^2 = x^2 - 2x + 1 \ge 0$.
We have $x^2 + 1 \ge 2x$ because ${(x-1)}^2 = x^2 - 2x + 1 \ge 0$.
We have $x + \frac{1}{x} \ge 2$ because $x^2 + 1 \ge 2x$ and $x>0$.
\end{tcolorbox}

Lean 4 code:
\begin{tcolorbox}[colback=white!10, width=\linewidth]
\begin{lstlisting}[language=Lean4]
import Mathlib
import Obvious
open Obvious

def h(x : ℝ)(h1 : x > 0) : x + 1 / x ≥ 2 := by
  have h2 : x > 0 := by obvious
  first
  | have h3 : (x - 1) ^ 2 ≥ 0 := by calc
    (x - 1) ^ 2 = x ^ 2 - 2 * x + 1 := by obvious
    _ ≥ 0 := by obvious
  | have h4 : x ^ 2 - 2 * x + 1 ≥ 0 := by calc
    x ^ 2 - 2 * x + 1 = (x - 1) ^ 2 := by obvious
    _ ≥ 0 := by obvious
  have h5 : x ^ 2 + 1 ≥ 2 * x := by obvious
  have h6 : x + 1 / x ≥ 2 := by obvious
  obvious

\end{lstlisting}
\end{tcolorbox}


\begin{example}
Problem:
\begin{tcolorbox}[colback=yellow!10, width=\linewidth]
For positive real numbers $a$ and $b$, prove:
    $$\left(\frac{a+b}{2}\right)^2 \leq \frac{a^2+b^2}{2}.$$
\end{tcolorbox}

Simplified proof:
\begin{tcolorbox}[colback=blue!10, width=\linewidth]
Since $(a-b)^2 \ge 0$, we have $a^2 - 2ab + b^2 \ge 0$, so $2ab \leq a^2 + b^2$.
Adding $a^2+b^2$ to both sides gives $a^2 + 2ab + b^2 \leq 2(a^2+b^2)$, so $(a+b)^2 \leq 2(a^2+b^2)$, and $\left(\frac{a+b}{2}\right)^2 \leq \frac{a^2+b^2}{2}$.
\end{tcolorbox}
\end{example}

Elaborated proof:
\begin{tcolorbox}[colback=green!10, width=\linewidth]
Since $(a-b)^2 \ge 0$, we have $a^2 - 2ab + b^2 \ge 0$, so $2ab \leq a^2 + b^2$.
Adding $a^2+b^2$ to both sides gives $a^2 + 2ab + b^2 \leq (a^2 + b^2) + 2ab \leq (a^2+b^2) + (a^2+b^2) = 2(a^2+b^2)$, so $(a+b)^2 \leq 2(a^2+b^2)$, and $\frac{(a+b)^2}{4} = \left(\frac{a+b}{2}\right)^2 \leq \frac{2(a^2+b^2)}{4} = \frac{a^2+b^2}{2}$.
\end{tcolorbox}

Regularized proof:
\begin{tcolorbox}[colback=red!10, width=\linewidth]
Let $a\in\mathbb{R}$.
Let $b\in\mathbb{R}$.
The goal is to prove ${\left(\frac{a+b}{2}\right)}^2 \leq \frac{a^2+b^2}{2}$.
We have ${(a-b)}^2 \ge 0$.
We have $a^2 - 2ab + b^2 \ge 0$.
We have $2ab \leq a^2 + b^2$.
We have $a^2 + 2ab + b^2 \leq (a^2 + b^2) + 2ab \leq (a^2+b^2) + (a^2+b^2) = 2(a^2+b^2)$.
We have ${(a+b)}^2 \leq 2(a^2+b^2)$.
We have $\frac{{(a+b)}^2}{4} = {\left(\frac{a+b}{2}\right)}^2 \leq \frac{2(a^2+b^2)}{4} = \frac{a^2+b^2}{2}$.
\end{tcolorbox}

Lean 4 code:
\begin{tcolorbox}[colback=white!10, width=\linewidth]
\begin{lstlisting}[language=Lean4]
import Mathlib
import Obvious
open Obvious

def h(a : ℝ)(b : ℝ) : ((a + b) / 2) ^ 2 ≤ (a ^ 2 + b ^ 2) / 2 := by
  have h1 : (a - b) ^ 2 ≥ 0 := by obvious
  have h2 : a ^ 2 - 2 * a * b + b ^ 2 ≥ 0 := by obvious
  have h3 : 2 * a * b ≤ a ^ 2 + b ^ 2 := by obvious
  have h4 : a ^ 2 + 2 * a * b + b ^ 2 ≤ 2 * (a ^ 2 + b ^ 2) := by calc
    a ^ 2 + 2 * a * b + b ^ 2 ≤ a ^ 2 + b ^ 2 + 2 * a * b := by obvious
    _ ≤ a ^ 2 + b ^ 2 + (a ^ 2 + b ^ 2) := by obvious
    _ = 2 * (a ^ 2 + b ^ 2) := by obvious
  have h5 : (a + b) ^ 2 ≤ 2 * (a ^ 2 + b ^ 2) := by obvious
  first
  | have h6 : (a + b) ^ 2 / 4 ≤ (a ^ 2 + b ^ 2) / 2 := by calc
    (a + b) ^ 2 / 4 = ((a + b) / 2) ^ 2 := by obvious
    _ ≤ 2 * (a ^ 2 + b ^ 2) / 4 := by obvious
    _ = (a ^ 2 + b ^ 2) / 2 := by obvious
  | have h7 : ((a + b) / 2) ^ 2 ≤ (a ^ 2 + b ^ 2) / 2 := by calc
    ((a + b) / 2) ^ 2 = (a + b) ^ 2 / 4 := by obvious
    _ ≤ 2 * (a ^ 2 + b ^ 2) / 4 := by obvious
    _ = (a ^ 2 + b ^ 2) / 2 := by obvious
  obvious

\end{lstlisting}
\end{tcolorbox}
\end{document}