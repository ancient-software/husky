%!TEX TS-program = xelatex
\documentclass{article}
\usepackage{amsmath}
\usepackage{amssymb}
\usepackage{fvextra}
\usepackage{tcolorbox}
\usepackage{listings}
\usepackage{amsthm}
\usepackage{fontspec}  % For Unicode support
\setmonofont{DejaVu Sans Mono}  % For monospace/code blocks
\usepackage{unicode-math}
% \setmathfont{XITS Math}  % Or another Unicode math font
\newtheorem{example}{Example}
\fvset{breaklines=true}



\begin{document}
\lstdefinelanguage{Lean4}{
    breaklines=true,
    basicstyle=\ttfamily\normalsize,
    keepspaces=true,
    morekeywords={
        def, theorem, lemma, example, have, show, calc, let, assume, by, exact,
        sorry, obvious, Type, Prop, where, with, extends, class, instance,
        structure, inductive, mutual, coinductive, variable, variables,
        universe, universes, deriving, abbrev, partial, terminating,
        namespace, import, open, export, private, protected, public,
        noncomputable, unsafe, macro, syntax, macro_rules, set_option,
        attribute, local, scoped, section, end, match, fun, if, then, else,
        return, do, for, in, while, break, continue, try, catch, throw,
        mut, pure, opaque
    },
    morekeywords=[2]{
        ℕ, ℝ, ∈, ≥, ≤, →, ∀, ∃, ⊢, ∧, ∨, ¬, ≠, ×, ⊗, ⊕, ∘, □, ◇, ∎,
        ⟨, ⟩, ⦃, ⦄, ▸, ≈, ∼, ≡, ⌊, ⌋, ⌈, ⌉
    },
    morecomment=[l]--,     % Line comments start with --
    morestring=[b]",       % Strings in double quotes
    sensitive=true,         % Case-sensitive
    keywordstyle=\color{blue}\bfseries,      % Regular keywords in blue and bold
    keywordstyle=[2]\color{purple}\bfseries, % Special symbols in purple and bold
    commentstyle=\color{green!50!black},     % Comments in dark green
    stringstyle=\color{red},                 % Strings in red
}



\begin{example}
Problem:
\begin{tcolorbox}[colback=yellow!10, width=\linewidth]
Prove: $1+1=2$.
\end{tcolorbox}

Simplified proof:
\begin{tcolorbox}[colback=blue!10, width=\linewidth]
It's trivial by the Peano axioms that $1+1=2$.
\end{tcolorbox}
\end{example}

Elaborated proof:
\begin{tcolorbox}[colback=green!10, width=\linewidth]
It's trivial by the Peano axioms that $1+1=2$.
\end{tcolorbox}

Regularized proof:
\begin{tcolorbox}[colback=red!10, width=\linewidth]
Let $x \in \mathbb{R}$.
Let $y \in \mathbb{R}$.
Assume $x=1$.
Assume $y=1$.
The goal is to prove $x+y=2$.
We have $1+1=2$ by the Peano axioms.
\end{tcolorbox}

Lean 4 code:
\begin{tcolorbox}[colback=white!10, width=\linewidth]
\begin{lstlisting}[language=Lean4]
import Mathlib
import Obvious
open Obvious

def h(x : ℝ)(y : ℝ)(h1 : x = 1)(h2 : y = 1) : x + y = 2 := by
  have h3 : 1 + 1 = 2 := by obvious
  obvious

\end{lstlisting}
\end{tcolorbox}
\end{document}