%!TEX TS-program = xelatex
\documentclass{article}
\usepackage{amsmath}
\usepackage{amssymb}
\usepackage{fvextra}
\usepackage{tcolorbox}
\usepackage{listings}
\usepackage{amsthm}
\usepackage{fontspec}  % For Unicode support
\setmonofont{DejaVu Sans Mono}  % For monospace/code blocks
\usepackage{unicode-math}
% \setmathfont{XITS Math}  % Or another Unicode math font
\newtheorem{example}{Example}
\fvset{breaklines=true}



\begin{document}
\lstdefinelanguage{Lean4}{
    breaklines=true,
    basicstyle=\ttfamily\normalsize,
    keepspaces=true,
    morekeywords={
        def, theorem, lemma, example, have, show, calc, let, assume, by, exact,
        sorry, obvious, Type, Prop, where, with, extends, class, instance,
        structure, inductive, mutual, coinductive, variable, variables,
        universe, universes, deriving, abbrev, partial, terminating,
        namespace, import, open, export, private, protected, public,
        noncomputable, unsafe, macro, syntax, macro_rules, set_option,
        attribute, local, scoped, section, end, match, fun, if, then, else,
        return, do, for, in, while, break, continue, try, catch, throw,
        mut, pure, opaque
    },
    morekeywords=[2]{
        ℕ, ℝ, ∈, ≥, ≤, →, ∀, ∃, ⊢, ∧, ∨, ¬, ≠, ×, ⊗, ⊕, ∘, □, ◇, ∎,
        ⟨, ⟩, ⦃, ⦄, ▸, ≈, ∼, ≡, ⌊, ⌋, ⌈, ⌉
    },
    morecomment=[l]--,     % Line comments start with --
    morestring=[b]",       % Strings in double quotes
    sensitive=true,         % Case-sensitive
    keywordstyle=\color{blue}\bfseries,      % Regular keywords in blue and bold
    keywordstyle=[2]\color{purple}\bfseries, % Special symbols in purple and bold
    commentstyle=\color{green!50!black},     % Comments in dark green
    stringstyle=\color{red},                 % Strings in red
}



\begin{example}
Problem:
\begin{tcolorbox}[colback=yellow!10, width=\linewidth]
Prove: $1+1=2$.
\end{tcolorbox}

Simplified proof:
\begin{tcolorbox}[colback=blue!10, width=\linewidth]
This is trivial.
\end{tcolorbox}
\end{example}

Elaborated proof:
\begin{tcolorbox}[colback=green!10, width=\linewidth]
This is trivial.
\end{tcolorbox}

Regularized proof:
\begin{tcolorbox}[colback=red!10, width=\linewidth]
The goal is to prove $1+1=2$.
We have $1+1=2$.
\end{tcolorbox}

Lean 4 code:
\begin{tcolorbox}[colback=white!10, width=\linewidth]
\begin{lstlisting}[language=Lean4]
import Mathlib
syntax "attack" : tactic

macro_rules
| `(tactic| attack) => `(tactic|
  first
  | simp; done
  | ring; done
  | ring_nf; rw [Real.sq_sqrt]; ring; all_goals attack; done
  | nlinarith; done
  | apply sq_nonneg; all_goals attack; done
  | apply div_nonneg; all_goals attack; done
  | field_simp; ring; done
  | linarith; done
)

macro "obvious": tactic =>`(tactic|
  first
  | attack; done
  | congr; all_goals attack; done
  | gcongr; all_goals attack; done
  | fail "Could not prove this goal automatically. It might not be as obvious as you think!"
)
def h : 1 + 1 = 2 := by
  have h1 : 1 + 1 = 2 := by obvious
  obvious

\end{lstlisting}
\end{tcolorbox}


\begin{example}
Problem:
\begin{tcolorbox}[colback=yellow!10, width=\linewidth]
Let $x\in\mathbb{R}$. Prove: $x^2\ge 0$.
\end{tcolorbox}

Simplified proof:
\begin{tcolorbox}[colback=blue!10, width=\linewidth]
The square of a real number is non-negative.
\end{tcolorbox}
\end{example}

Elaborated proof:
\begin{tcolorbox}[colback=green!10, width=\linewidth]
The square of a real number is non-negative.
\end{tcolorbox}

Regularized proof:
\begin{tcolorbox}[colback=red!10, width=\linewidth]
Let $x\in\mathbb{R}$.
The goal is to prove ${{x}}^2 \ge 0$.
We have ${{x}}^2 \ge 0$.
\end{tcolorbox}

Lean 4 code:
\begin{tcolorbox}[colback=white!10, width=\linewidth]
\begin{lstlisting}[language=Lean4]
import Mathlib
syntax "attack" : tactic

macro_rules
| `(tactic| attack) => `(tactic|
  first
  | simp; done
  | ring; done
  | ring_nf; rw [Real.sq_sqrt]; ring; all_goals attack; done
  | nlinarith; done
  | apply sq_nonneg; all_goals attack; done
  | apply div_nonneg; all_goals attack; done
  | field_simp; ring; done
  | linarith; done
)

macro "obvious": tactic =>`(tactic|
  first
  | attack; done
  | congr; all_goals attack; done
  | gcongr; all_goals attack; done
  | fail "Could not prove this goal automatically. It might not be as obvious as you think!"
)
def h (x : ℝ) : x ^ 2 ≥ (0 : ℝ) := by
  have h1 : x ^ 2 ≥ (0 : ℝ) := by obvious
  obvious

\end{lstlisting}
\end{tcolorbox}


\begin{example}
Problem:
\begin{tcolorbox}[colback=yellow!10, width=\linewidth]
Let $a\in\mathbb{R}$. Let $b\in\mathbb{R}$. Prove: $a+b=b+a$.
\end{tcolorbox}

Simplified proof:
\begin{tcolorbox}[colback=blue!10, width=\linewidth]
Trivial by the commutativity of addition for real numbers.
\end{tcolorbox}
\end{example}

Elaborated proof:
\begin{tcolorbox}[colback=green!10, width=\linewidth]
Trivial by the commutativity of addition for real numbers.
\end{tcolorbox}

Regularized proof:
\begin{tcolorbox}[colback=red!10, width=\linewidth]
Let $a\in\mathbb{R}$.
Let $b\in\mathbb{R}$.
The goal is to prove $a+b=b+a$.
We have $a+b=b+a$ by the commutativity of addition for real numbers.
\end{tcolorbox}

Lean 4 code:
\begin{tcolorbox}[colback=white!10, width=\linewidth]
\begin{lstlisting}[language=Lean4]
import Mathlib
syntax "attack" : tactic

macro_rules
| `(tactic| attack) => `(tactic|
  first
  | simp; done
  | ring; done
  | ring_nf; rw [Real.sq_sqrt]; ring; all_goals attack; done
  | nlinarith; done
  | apply sq_nonneg; all_goals attack; done
  | apply div_nonneg; all_goals attack; done
  | field_simp; ring; done
  | linarith; done
)

macro "obvious": tactic =>`(tactic|
  first
  | attack; done
  | congr; all_goals attack; done
  | gcongr; all_goals attack; done
  | fail "Could not prove this goal automatically. It might not be as obvious as you think!"
)
def h (a b : ℝ) : a + b = b + a := by
  have h1 : a + b = b + a := by obvious
  obvious

\end{lstlisting}
\end{tcolorbox}


\begin{example}
Problem:
\begin{tcolorbox}[colback=yellow!10, width=\linewidth]
Let $x\in\mathbb{R}$. Assume $x> 0$. Prove: $x + \frac{1}{x} \ge 2$.
\end{tcolorbox}

Simplified proof:
\begin{tcolorbox}[colback=blue!10, width=\linewidth]
Since $x>0$, we have $(x-1)^2 \ge 0$. Thus, $x - 2 + \frac{1}{x} \ge 0$, so $x + \frac{1}{x} \ge 2$.
\end{tcolorbox}
\end{example}

Elaborated proof:
\begin{tcolorbox}[colback=green!10, width=\linewidth]
Since $x>0$, we have $(x-1)^2 = x^2 - 2x + 1 \ge 0$. Thus, $x^2 - 2x + 1 \ge 0$, so $x^2 + 1 \ge 2x$. Since $x > 0$, dividing both sides by $x$ gives $x + \frac{1}{x} \ge 2$.
\end{tcolorbox}

Regularized proof:
\begin{tcolorbox}[colback=red!10, width=\linewidth]
Let $x\in\mathbb{R}$.
Assume $x>0$.
The goal is to prove $x + \frac{1}{x} \ge 2$.
We have $x>0$.
We have ${{(x-1)}}^2 = x^2 - 2x + 1 \ge 0$.
We have $x^2 - 2x + 1 \ge 0$.
We have $x^2 + 1 \ge 2x$.
We have $x > 0$.
We have $x + \frac{1}{x} \ge 2$.
\end{tcolorbox}

Lean 4 code:
\begin{tcolorbox}[colback=white!10, width=\linewidth]
\begin{lstlisting}[language=Lean4]
import Mathlib
syntax "attack" : tactic

macro_rules
| `(tactic| attack) => `(tactic|
  first
  | simp; done
  | ring; done
  | ring_nf; rw [Real.sq_sqrt]; ring; all_goals attack; done
  | nlinarith; done
  | apply sq_nonneg; all_goals attack; done
  | apply div_nonneg; all_goals attack; done
  | field_simp; ring; done
  | linarith; done
)

macro "obvious": tactic =>`(tactic|
  first
  | attack; done
  | congr; all_goals attack; done
  | gcongr; all_goals attack; done
  | fail "Could not prove this goal automatically. It might not be as obvious as you think!"
)
def h (x : ℝ) (h1 : x > (0 : ℝ)) : x + (1 : ℝ) / x ≥ (2 : ℝ) := by
  have h2 : x > (0 : ℝ) := by obvious
  first
  | have h3 : (x - (1 : ℝ)) ^ 2 ≥ (0 : ℝ) := by calc
    (x - (1 : ℝ)) ^ 2 = x ^ 2 - (2 : ℝ) * x + (1 : ℝ) := by obvious
    _ ≥ (0 : ℝ) := by obvious
  | have h4 : x ^ 2 - (2 : ℝ) * x + (1 : ℝ) ≥ (0 : ℝ) := by calc
    x ^ 2 - (2 : ℝ) * x + (1 : ℝ) = (x - (1 : ℝ)) ^ 2 := by obvious
    _ ≥ (0 : ℝ) := by obvious
  have h5 : x ^ 2 - (2 : ℝ) * x + (1 : ℝ) ≥ (0 : ℝ) := by obvious
  have h6 : x ^ 2 + (1 : ℝ) ≥ (2 : ℝ) * x := by obvious
  have h7 : x > (0 : ℝ) := by obvious
  have h8 : x + (1 : ℝ) / x ≥ (2 : ℝ) := by obvious
  obvious

\end{lstlisting}
\end{tcolorbox}


\begin{example}
Problem:
\begin{tcolorbox}[colback=yellow!10, width=\linewidth]
Let $x\in\mathbb{R}$. Prove: $x^2 + 1\ge 2x$.
\end{tcolorbox}

Simplified proof:
\begin{tcolorbox}[colback=blue!10, width=\linewidth]
$(x-1)^2 \ge 0$, so $x^2 + 1 \ge 2x$.
\end{tcolorbox}
\end{example}

Elaborated proof:
\begin{tcolorbox}[colback=green!10, width=\linewidth]
$(x-1)^2 = x^2 - 2x + 1 \ge 0$, so $x^2 + 1 \ge 2x$.
\end{tcolorbox}

Regularized proof:
\begin{tcolorbox}[colback=red!10, width=\linewidth]
Let $x\in\mathbb{R}$.
The goal is to prove $x^2 + 1 \ge 2x$.
We have ${{x-1}}^2 = x^2 - 2x + 1 \ge 0$.
We have $x^2 + 1 \ge 2x$.
\end{tcolorbox}

Lean 4 code:
\begin{tcolorbox}[colback=white!10, width=\linewidth]
\begin{lstlisting}[language=Lean4]
import Mathlib
syntax "attack" : tactic

macro_rules
| `(tactic| attack) => `(tactic|
  first
  | simp; done
  | ring; done
  | ring_nf; rw [Real.sq_sqrt]; ring; all_goals attack; done
  | nlinarith; done
  | apply sq_nonneg; all_goals attack; done
  | apply div_nonneg; all_goals attack; done
  | field_simp; ring; done
  | linarith; done
)

macro "obvious": tactic =>`(tactic|
  first
  | attack; done
  | congr; all_goals attack; done
  | gcongr; all_goals attack; done
  | fail "Could not prove this goal automatically. It might not be as obvious as you think!"
)
def h (x : ℝ) : x ^ 2 + (1 : ℝ) ≥ (2 : ℝ) * x := by
  first
  | have h1 : (x - (1 : ℝ)) ^ 2 ≥ (0 : ℝ) := by calc
    (x - (1 : ℝ)) ^ 2 = x ^ 2 - (2 : ℝ) * x + (1 : ℝ) := by obvious
    _ ≥ (0 : ℝ) := by obvious
  | have h2 : x ^ 2 - (2 : ℝ) * x + (1 : ℝ) ≥ (0 : ℝ) := by calc
    x ^ 2 - (2 : ℝ) * x + (1 : ℝ) = (x - (1 : ℝ)) ^ 2 := by obvious
    _ ≥ (0 : ℝ) := by obvious
  have h3 : x ^ 2 + (1 : ℝ) ≥ (2 : ℝ) * x := by obvious
  obvious

\end{lstlisting}
\end{tcolorbox}
\end{document}