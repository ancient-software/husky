\documentclass[12pt]{article}
\usepackage{amsmath}
\usepackage{amssymb}
\usepackage{amsthm}
\usepackage{natbib}
\newtheorem{example}{Example}
\begin{document}


%%%%%% Note: I just skip all the requirement of x > 0, because it seems that adding 'Assume $x>0$.' will has some problem.

%%%%%%%%% success examples %%%%%%%%%
\begin{example}
    Then $1+1=2$.
\end{example}

\begin{example}
    Let $x\in\mathbb{R}$. Then $x^2\ge 0$.
\end{example}

\begin{example}
    Let $a\in\mathbb{R}$. Let $b\in\mathbb{R}$. Then $a+b=b+a$.
\end{example}

\begin{example}
    Let $x\in\mathbb{R}$. Then $x^2 \ge 0$.
\end{example}

% prove x + 1/x \geq 2 
% fail case:
% TODO: when I has two eq in a $$, like $x + \frac{1}{x} - 2 = \frac{x^2 + 1 - 2x}{x} = \frac{{(x-1)}^2}{x}$.
\begin{example}
    Let $x\in\mathbb{R}$. Assume $x>0$. Then $x + \frac{1}{x} - 2 = \frac{x^2 + 1 - 2x}{x}$.
    Then $\frac{x^2 + 1 - 2x}{x} = \frac{{(x-1)}^2}{x}$.
    Then $x + \frac{1}{x} - 2 = \frac{{(x-1)}^2}{x}$.
    Then $\frac{{(x-1)}^2}{x} \ge 0$.
    Then $x + \frac{1}{x} - 2 \ge 0$.
    Then $x + \frac{1}{x} \ge 2$.
\end{example}
\begin{example}
    Let $x\in\mathbb{R}$. Then $x + \frac{1}{x} - 2 = \frac{x^2 + 1 - 2x}{x}$.
    Then $\frac{x^2 + 1 - 2x}{x} = \frac{{(x-1)}^2}{x}$.
    Then $x + \frac{1}{x} - 2 = \frac{{(x-1)}^2}{x}$.
    Then $\frac{{(x-1)}^2}{x} \ge 0$.
    Then $x + \frac{1}{x} - 2 \ge 0$.
    Then $x + \frac{1}{x} \ge 2$.
\end{example}



%%%%%%%%% failure examples %%%%%%%%%
% prove x^2 + 1 \geq 2x
\begin{example}
    Let $x\in\mathbb{R}$. Then $x^2 + 1 - 2x = {(x-1)}^2$. Then ${(x-1)}^2 \ge 0$. Then $x^2 + 1 - 2x \ge 0$. Then $x^2 + 1 \ge 2x$.
\end{example}

% prove x + 1 \geq 2\sqrt{x}
\begin{example}
    Let $x\in\mathbb{R}$. Assume $x > 0$. Then $x + 1 - 2\sqrt{x} = {(\sqrt{x}-1)}^2$. Then ${(\sqrt{x}-1)}^2 \ge 0$. Then $x + 1 - 2\sqrt{x} \ge 0$. Then $x + 1 \ge 2\sqrt{x}$.
\end{example}

% % prove 1/x + 1/y \geq 4/(x+y)
\begin{example}
    Let $x\in\mathbb{R}$. Let $y\in\mathbb{R}$.
    Then $\frac{1}{x} + \frac{1}{y} - \frac{4}{x+y} = \frac{y(x+y) + x(x+y) - 4xy}{xy(x+y)}$.
    Then $\frac{y(x+y) + x(x+y) - 4xy}{xy(x+y)} = \frac{yx + x^2 + x^2 + yx - 4xy}{xy(x+y)}$.
    Then $\frac{yx + x^2 + x^2 + yx - 4xy}{xy(x+y)} = \frac{x^2 + x^2 -2xy}{xy(x+y)}$. % may has a jump step here
    Then $\frac{x^2 + x^2 -2xy}{xy(x+y)} = \frac{{(x-y)}^2}{xy(x+y)}$.
    Then $\frac{{(x-y)}^2}{xy(x+y)} \ge 0$.
    Then $\frac{1}{x} + \frac{1}{y} - \frac{4}{x+y} \ge 0$.
    Then $\frac{1}{x} + \frac{1}{y} \ge \frac{4}{x+y}$.
\end{example}

% prove a/b + b/a \geq 2
\begin{example}
    Let $a\in\mathbb{R}$. Let $b\in\mathbb{R}$.
    Then $\frac{a}{b} + \frac{b}{a} - 2 = \frac{a^2 + b^2 - 2ab}{ab}$.
    Then $\frac{a^2 + b^2 - 2ab}{ab} = \frac{{(a-b)}^2}{ab}$.
    Then $\frac{{(a-b)}^2}{ab} \ge 0$.
    Then $\frac{a}{b} + \frac{b}{a} - 2 \ge 0$.
    Then $\frac{a}{b} + \frac{b}{a} \ge 2$.
\end{example}

% prove x^2 + y^2 \ge 2xy
\begin{example}
    Let $x\in\mathbb{R}$. Let $y\in\mathbb{R}$.
    Then $x^2 + y^2 - 2xy = {(x-y)}^2$.
    Then ${(x-y)}^2 \ge 0$.
    Then $x^2 + y^2 - 2xy \ge 0$.
    Then $x^2 + y^2 \ge 2xy$.
\end{example}

\end{document}
